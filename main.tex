\documentclass[a4paper]{article}

\usepackage[utf8]{inputenc}
\usepackage[T1]{fontenc}
\usepackage[french]{babel}
\usepackage{fullpage}
\usepackage{hyperref}
\usepackage{amsmath}
\usepackage{amssymb}
\usepackage{upgreek}
\usepackage{color}
\usepackage[]{algorithm2e}
\usepackage{stmaryrd}
\usepackage{graphicx}
\usepackage{float}

\title{
    ELEC-H-201 - Electronique\\
    \small Synthèse 2014-2015
}
\author{Florentin \bsc{Hennecker}}
\date{}

\begin{document}
\maketitle
\tableofcontents

\section{Vademecum d'électricité (chapitre 2)}

    \paragraph{Schémas}
    \begin{itemize}
        \item La \textit{charge} d'un montage est l'équipement/composant en aval d'un montage
        \item La \textit{source} d'un montage est l'équipement/composant en amont d'un montage
        \item Deux composants sont connectés en \textit{série} s'ils sont parcourus par le même courant
        \item Deux composants sont connectés en \textit{parallèle} s'ils sont soumis à la même ddp
    \end{itemize}

    \paragraph{Courants et tensions}
    \begin{itemize}
        \item L'\textit{intensité} est le "débit" de charge électrique : $i() = \frac{dq(t)}{dt}$
    \end{itemize}

\end{document}